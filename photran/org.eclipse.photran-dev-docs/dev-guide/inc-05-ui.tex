% The Photran UI Plug-in

The Photran UI plug-in (org.eclipse.photran.ui) contains the Fortran editor
and several preference pages.

Eclipse editors have a very non-intuitive structure which is very nicely
explained elsewhere (for example, in \textit{The Java Developer's Guide
to Eclipse}).

\section{Lexer-based Syntax Highlighting}

The main difference between our Fortran editor and a ``normal'' Eclipse
editor is that we do not use the typical means of syntax highlighting.
Since Fortran does not have reserved words, keywords such as ``if'' and
``do'' can also be used as identifiers.  So the word ``if'' may need
to be highlighted as a keyword in one case and as a varible in another.

To do this, we actually run the Fortran lexical analyzer over the entire
source file.  It splits the input into tokens and specifies whether they
are identifiers or not.  We create a partition for each token.  We also
create a partition for the space between tokens.  Each entire partition,
then, is given a single color, based on its contents (keyword, identifier,
or comments/whitespace).  This is all done in the class
\texttt{FortranPartitionScanner}.
