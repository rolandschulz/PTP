% The Photran Core Plug-in

The Photran Core plug-in (org.eclipse.photran.core) contains several source
folders:

\begin{itemize}
\item \texttt{src} contains the main plug-in class
(\texttt{FortranCorePlugin}), the Fortran implementation of
\texttt{IAdditionalLanguage} (\texttt{FortranLanguage}), and any other
``miscellaneous'' classes that don't fit into one of the other folders.
\item \texttt{errorparsers} contains a number of built-in error parsers
for popular Fortran compilers.
\item \texttt{model} contains the model builder for Fortran
(\texttt{FortranModelBuilder}) and the Fortran model elements
(\texttt{FortranElement} and nested subclasses).
\item \texttt{parser} contains the \texttt{FortranProcessor} class,
which is an interface to the parser, which is stored in
\texttt{f95parser.jar}.  It also contains all of the symbol table classes.
\item \texttt{refactoring} contains everything related to refactoring
that is not used elsewhere in Photran, for example, the
\texttt{Program} and \texttt{Presentation} classes, the program editor,
the source printer, etc.  (All of these are described later.)
\item \texttt{preferences} contains classes which ``wrap'' the various
preferences that can be set in the Core plug-in.  The actual preference
pages displayed to the user are, of course, in the UI plug-in, but they
use these classes to get and set the preference values.
\end{itemize}

The parser JAR (\texttt{f95parser.jar}) is contained in the root folder
of this plug-in.

\section{Error Parsers}
Error parsers scan the output of \texttt{make} for error messages
for a particular compiler.
When they see an error message they can recognize, they extract the
filename, line number, and error message, and use it to populate the
Problems view.

For an example, see \texttt{IntelFortranErrorParser}.
(It's a mere 74 lines.)

To create a new error parser,
\begin{itemize}

\item In package \texttt{org.eclipse.photran.internal.errorparsers},
  define a class implementing \texttt{IErrorParser}

\item Implement \texttt{public boolean processLine(String line,
ErrorParserManager eoParser)}
  which should always returns false because ErrorParserManager appears not to
  use the result in a rational way

\item In org.eclipse.photran.core's \texttt{plugin.xml}, find the place
where we define all of the Fortran error parsers.  Basically, copy an
existing one.  Your addition will look something like this:
\begin{verbatim}
   <extension
         id="IntelFortranErrorParser"
         name="Photran Error Parser for Some New Fortran Compiler"
         point="org.eclipse.cdt.core.ErrorParser">
      <errorparser
            class="org.eclipse.photran.internal.errorparsers.MyErrorParser">
      </errorparser>
   </extension>
\end{verbatim}

\item Your new error parser will appear in the error parser list in the
Preferences automatically, and it will be automatically added to new projects.
For existing projects, you will need to open the project properties dialog
and add the new error parser to the project manually.

\end{itemize}

\textbf{Note.}  Error parsers to not have to be implemented in the Photran
Core plug-in.  In fact, they do not have to be implemented in Photran at all.
If you create a brand new plug-in, you can specify org.eclipse.cdt.core
as a dependency, include the above XML snippet in your plug-in's
\texttt{plugin.xml}, and include your custom error parser class in that
plug-in.
